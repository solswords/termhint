\documentclass[submission,copyright,creativecommons]{eptcs}
\providecommand{\event}{ACL2 Workshop 2017} % Name of the event you are submitting to
% \usepackage{breakurl}             % Not needed if you use pdflatex only.
\usepackage{amsmath,amssymb,amsfonts,amsthm}
\usepackage{fancyvrb}
% so we can use \-/ for breakable dashes in long names
\usepackage[shortcuts]{extdash}
% \usepackage{bigfoot}
% \usepackage{tikz}
\usepackage{color}
\usepackage{framed}

% \newtheorem{theorem}{Theorem}
% \newtheorem{corollary}{Corollary}

% The first of the following two definitions of the comment
% environment is adapted from
% http://tex.stackexchange.com/questions/71371/how-to-change-color-in-an-environment
% while the second comes from a package.  Choose the first to see
% comments and the second to hide them.

\newenvironment{comment}
  {\par\medskip
   \color{red}%
   \begin{framed}
   \ignorespaces}
 {\end{framed}
  \medskip}

% \usepackage{comment}

% Neither of the following worked for me:

%\newenvironment{mycomment}{{\begin{comment}}}{{\end{comment}}}

% \let\mycomment\comment
% \let\endmycomment\endcomment

\usepackage{listings}
\lstset{ %
  basicstyle=\normalsize\ttfamily,
  language=lisp,
  columns=fullflexible,
  escapeinside={\<}{\>},
}

\newcommand*{\var}[1]{\mathit{#1}}

\title{Hint Orchestration Using ACL2's Simplifier}
\author{Sol Swords
\institute{Centaur Techology, Inc.\\
Austin, TX, USA}
\email{sswords@centtech.com}
}
\def\titlerunning{Hint Orchestration}
\def\authorrunning{S. Swords}
\begin{document}
\maketitle

\begin{abstract}
  This paper describes a mechanism for providing hints during a proof.
  An extra literal is added to the goal clause and simplified along
  with the rest of the goal until it is stable under simplification,
  after which the simplified literal is examined and a hint extracted
  from it.  This simple technique supports some commonly desirable yet
  elusive features: it can provide different hints to different cases
  of a case split; it can share subterms among the terms supplied in a
  hint; and since the terms used in the hint have been simplified
  along with the rest of the goal, they match other occurrences of
  those terms in the rest of the goal.  We additionally describe an
  extension to this mechanism that can sequence such hints.

\end{abstract}


ACL2's mechanism for giving hints to the prover is very powerful and
general.  Users can provide arbitrary code to evaluate whether a hint
should be given and what that hint should be.  However, it is
surprisingly difficult to accomplish certain things with computed
hints.  Consider a proof that splits into two cases, one assuming term
\texttt{A} true and one assuming it false.  Suppose we want to supply
a hint \texttt{H1} when \texttt{A} is assumed true and
\texttt{H2} when it is assumed false.  A computed hint can examine the
clause to see when the case split occurs and which case has resulted
-- if \texttt{(not A)} is a member of the clause, then \texttt{A} is
assumed true, and if \texttt{A} is a member of the clause, it is
assumed false:
\begin{verbatim}
  :hints ((and (member-equal '(not A) clause) H1)
          (and (member-equal 'A clause) H2))
\end{verbatim}
However, this strategy is more brittle than it may appear.  A small
change in rewriting strategy could cause the exact form of the term
\texttt{A} in the clause to change, even if the proof still case
splits on a term with the same meaning.  Even a change to inlining of
a function can affect the macroexpansion of the term, which changes
the exact appearance of the term in the clause.




  When a proof splits
into cases, often one wants to give different hints in different
conditions.  Explicit hints are a poor solution for this because
they are broken whenever subgoal numbers change.  Computed hints are
general enough to accomplish this, but 


  Explicit hints
  (except when triggered on ``Goal'') can be broken by any change in
  subgoal numbering, which happens often due to changes in underlying
  function definitions or system heuristics.  Computed hints are a
  very general mechanism that can be difficult to harness effectively
  without intermediate utilities.  In this paper we discuss a way to use computed hints to 


% \nocite{*}
\bibliographystyle{eptcs}
\bibliography{paper}
\end{document}


